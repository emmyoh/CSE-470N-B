\lesson{Tue, 11 April 2023, 2:50pm -- 4:10pm}{Week 12, Tuesday}

In Figure~\ref{fig:lec13fig1}, we saw that the circuit for Deutsch's algorithm has four `slices', A, B, C, and D. Slice A is fairly straightforward, $\ket{01}$, being the inputs, and slice D is also fairly straightforward, either $\ket{0}$ or $\ket{1}$, being the possible outputs.

The state of the system at B is $\frac{1}{\sqrt{2}} (\ket{0} + \ket{1}) \otimes \frac{1}{\sqrt{2}} (\ket{0} - \ket{1}) = \frac{1}{2} (\ket{00} - \ket{01} + \ket{10} - \ket{11})$. The state of the system at C is $F(\frac{1}{2} (\ket{00} - \ket{01} + \ket{10} - \ket{11})) = \frac{1}{2}(F(\ket{00}) - F(\ket{01}) + F(\ket{10}) - F(\ket{11})) = \frac{1}{2}(\ket{0} \otimes (0 \oplus f(0)) - \ket{0} \otimes (1 \oplus f(0))) + (\ket{1} \otimes (0 \oplus f(1)) - \ket{1} \otimes (1 \oplus f(1))) = \frac{1}{2}(\ket{0} \otimes f(0) - \ket{0} \otimes \overline{f(0)} + \ket{1} \otimes f(1) - \ket{1} \otimes \overline{f(1)})$.

The state of the system at C is therefore a superposition of the four possible outputs of the function, $f$:\\

Case 1 ($f(0) = 0$, $f(1) = 0$):\\
$\frac{1}{2}(\ket{00} - \ket{01} + \ket{10} - \ket{11}) = \frac{1}{\sqrt{2}} (\ket{0} + \ket{1}) \otimes \frac{1}{\sqrt{2}} (\ket{0} - \ket{1})$\\

Case 2 ($f(0) = 1$, $f(1) = 1$):\\
$\frac{1}{2}(\ket{01} - \ket{00} + \ket{11} - \ket{10}) = -\frac{1}{2} (\ket{00} - \ket{01} + \ket{10} - \ket{11}) = -\frac{1}{\sqrt{2}} (\ket{0} + \ket{1}) \otimes \frac{1}{\sqrt{2}} (\ket{0} - \ket{1})$\\

Case 3 ($f(0) = 0$, $f(1) = 1$):\\
$\frac{1}{2}(\ket{00} - \ket{01} + \ket{11} - \ket{10}) = \frac{1}{\sqrt{2}} (\ket{0} - \ket{1}) \otimes \frac{1}{\sqrt{2}} (\ket{0} - \ket{1})$\\

Case 4 ($f(0) = 1$, $f(1) = 0$):\\
$\frac{1}{2}(\ket{01} - \ket{00} + \ket{10} - \ket{11}) = -\frac{1}{2} (\ket{00} - \ket{01} + \ket{10} - \ket{11}) = -\frac{1}{\sqrt{2}} (\ket{0} + \ket{1}) \otimes \frac{1}{\sqrt{2}} (\ket{0} - \ket{1})$\\

For Case 1, D can be evaluated as $H(\frac{1}{\sqrt{2}}\ket{0} + \frac{1}{\sqrt{2}}\ket{1}) = \ket{0}$. For Case 2, D can be evaluated as $H(-\frac{1}{\sqrt{2}}\ket{0} + \frac{1}{\sqrt{2}}\ket{1}) = -\ket{0}$. For Case 3, D can be evaluated as $H(\frac{1}{\sqrt{2}}\ket{0} - \frac{1}{\sqrt{2}}\ket{1}) = \ket{1}$. For Case 4, D can be evaluated as $H(-\frac{1}{\sqrt{2}}\ket{0} - \frac{1}{\sqrt{2}}\ket{1}) = -\ket{1}$.

\begin{definition}
    The \emph{Deutsch-Jozsa algorithm} is a quantum algorithm that solves the following problem:\\

    Given a function $f: \{0,1\}^n \rightarrow \{0,1\}$, that is either constant or balanced, determine which it is.\\

    If $n=4$, the problem requires $9$ function calls in the worst case. If $n=3$, the problem requires $5$ function calls in the worst case. If $n=2$, the problem requires $3$ function calls in the worst case. This relationship is modeled by the expression $2^{n-1} + 1$. The Deutsch-Jozsa algorithm requires $1$ function call; it operates in constant time.
\end{definition}

\begin{definition}
    The $H^{\otimes 2}$ gate is $\frac{1}{2} \begin{bmatrix}
        1 & 1 & 1 & 1\\
        1 & -1 & 1 & -1\\
        1 & 1 & -1 & -1\\
        1 & -1 & -1 & 1
    \end{bmatrix} = \frac{1}{\sqrt{2}} \begin{bmatrix}
        H & H\\
        H & -H\\
    \end{bmatrix}$.

    It functions as follows:\\

    $H(\ket{0}) \otimes H(\ket{0}) = \frac{1}{\sqrt{2}} (\ket{0} + \ket{1}) \otimes \frac{1}{\sqrt{2}} (\ket{0} + \ket{1}) = \frac{1}{2} (\ket{00} + \ket{01} + \ket{10} + \ket{11}) = \frac{1}{2} \begin{bmatrix}
        1\\
        1\\
        1\\
        1
    \end{bmatrix}$\\
    
    $H(\ket{0}) \otimes H(\ket{1}) = \frac{1}{2} (\ket{00} - \ket{01} + \ket{10} - \ket{11}) = \frac{1}{2} \begin{bmatrix}
        1\\
        -1\\
        1\\
        -1
    \end{bmatrix}$\\
    
    $H(\ket{1}) \otimes H(\ket{0}) = \frac{1}{\sqrt{2}} (\ket{0} - \ket{1}) \otimes \frac{1}{\sqrt{2}} (\ket{0} + \ket{1}) = \frac{1}{2} (\ket{00} + \ket{01} - \ket{10} - \ket{11}) = \frac{1}{2} \begin{bmatrix}
        1\\
        1\\
        -1\\
        -1
    \end{bmatrix}$\\
    
    $H(\ket{1}) \otimes H(\ket{1}) = \frac{1}{2} (\ket{00} - \ket{10}) \otimes \frac{1}{2}(\ket{0} - \ket{1}) = \frac{1}{2}(\ket{00} - \ket{01} - \ket{10} + \ket{11}) = \frac{1}{2} \begin{bmatrix}
        1\\
        -1\\
        -1\\
        1
    \end{bmatrix}$\\
\end{definition}

\begin{note}
    $H^{\otimes 3} = \frac{1}{\sqrt{2}} \begin{bmatrix}
        H^{\otimes 2} & H^{\otimes 2}\\
        H^{\otimes 2} & -H^{\otimes 2}\\
    \end{bmatrix}$.
\end{note}

\begin{figure}[ht]
    \centering
    \begin{quantikz}
        \lstick{$\ket{00}$\\$x$}\slice{A} & \gate{H^{\otimes 2}}\slice{B} & \gate[wires=2]{F}\slice{C} & \gate{H^{\otimes 2}}\slice{D} & \meter{} & \qw \\
        \lstick{$\ket{1}$\\$y$} & \gate{H} & \qw & \qw & \qw & \qw
    \end{quantikz}
    \caption{The circuit for the Deutsch-Jozsa algorithm when $n = 2$, where $F(x_1, x_2, y) = \ket{x_1 x_2} \otimes (y \oplus f(x_1, x_2))$. At D, the state can be either $\ket{00}$, $\ket{01}$, $\ket{10}$, or $\ket{11}$.}\label{fig:lec14fig1}
\end{figure}

At slice A, the state is $\ket{001}$. At slice B, the state is $\frac{1}{\sqrt{2}} \begin{bmatrix}
    1 & 1 & 1 & 1\\
    1 & -1 & 1 & -1\\
    1 & 1 & -1 & -1\\
    1 & -1 & -1 & 1
\end{bmatrix} \cdot \begin{bmatrix}
    1\\
    0\\
    0\\
    0
\end{bmatrix} = \frac{1}{2} \begin{bmatrix}
    1\\
    1\\
    1\\
    1
\end{bmatrix} = \frac{1}{2} (\ket{00} + \ket{01} + \ket{10} + \ket{11})$. Expressed differently, $\frac{1}{2}(\ket{00} + \ket{01} + \ket{10} + \ket{11}) \otimes \frac{1}{\sqrt{2}} (\ket{0} - \ket{1}) = \frac{1}{2\sqrt{2}} (\ket{000} - \ket{001} + \ket{010} - \ket{011} + \ket{100} - \ket{101} + \ket{110} - \ket{111})$. At slice C, the state is $F(\frac{1}{2\sqrt{2}} (\ket{000} - \ket{001} + \ket{010} - \ket{011} + \ket{100} - \ket{101} + \ket{110} - \ket{111})) = \frac{1}{2\sqrt{2}} (F(\ket{000}) - F(\ket{001}) + F(\ket{010}) - F(\ket{011}) + F(\ket{100}) - F(\ket{101}) + F(\ket{110}) - F(\ket{111})) = \frac{1}{2\sqrt{2}} (\ket{00} \otimes (0 \oplus f(0)) - \ket{00} \otimes (1 \oplus f(00)) + \ket{01} \otimes (0 \oplus f(01)) - \ket{01} \otimes (1 \oplus f(01)) + \ket{10} \otimes (0 \oplus f(10)) - \ket{10} \otimes (1 \oplus f(10)) + \ket{11} \otimes (0 \oplus f(11)) - \ket{11} \otimes (1 \oplus f(11))) = \frac{1}{2\sqrt{2}}(\ket{00} \otimes f(00) - \ket{00} \otimes \overline{f(00)} + \ket{01 \otimes f(01) - \ket{01} \otimes \overline{f(01)} + \ket{01} \otimes f(10)} - \ket{10} \otimes \overline{f(10)} + \ket{11} \otimes f(11) - \ket{11} \otimes \overline{f(11)})$.

\begin{figure}[ht]
    \centering
    \begin{quantikz}
        \lstick{$\ket{0}^n$\\$x$}\slice{A} & \gate{H^{\otimes n}}\slice{B} & \gate[wires=2]{F}\slice{C} & \gate{H^{\otimes n}}\slice{D} & \meter{} & \qw \\
        \lstick{$\ket{1}$\\$y$} & \gate{H} & \qw & \qw & \qw & \qw
    \end{quantikz}
    \caption{The general circuit for the Deutsch-Jozsa algorithm.}\label{fig:lec14fig2}
\end{figure}

% \subsection{Sub Section 3}
% \label{sub_sec:sub_section_3}

% \begin{theorem}
% This is a theorem.
% \end{theorem}
% \begin{proof}
% This is a proof.
% \end{proof}
% \begin{example}
% This is an example.
% \end{example}
% \begin{explanation}
% This is an explanation.
% \end{explanation}
% \begin{claim}
% This is a claim.
% \end{claim}
% \begin{corollary}
% This is a corollary.
% \end{corollary}
% \begin{prop}
% This is a proposition.
% \end{prop}
% \begin{lemma}
% This is a lemma.
% \end{lemma}
% \begin{question}
% This is a question.
% \end{question}
% \begin{solution}
% This is a solution.
% \end{solution}
% \begin{exercise}
% This is an exercise.
% \end{exercise}
% \begin{definition}[Definition]
% This is a definition.
% \end{definition}
% \begin{note}
% This is a note.
% \end{note}

% subsection sub_section_2 (end)

\newpage