\lesson{Tue, 7 February 2023, 2:50pm – 4:10pm}{Week 3, Tuesday}

`The state of a quantum system is a vector in a vector space of complex numbers.'

Using the following basis vectors, $\ket{↑} = \ket{0} = \begin{bmatrix}
    1\\
    0
\end{bmatrix}$, $\ket{↓} = \ket{1} = \begin{bmatrix}
    0\\
    1
\end{bmatrix}$, a quantum state can be represented as $a\ket{0} + b\ket{1}$, where $a$ and $b$ are probability amplitudes (ie, $a^2$ is the probability of getting $\ket{0}$, $b^2$ is the probability of getting $\ket{1}$, and $a^2 + b^2 = 1$). To determine if a quantum state is valid, square the values of $a$ and $b$, and then add them together; if the sum of the squares is not equal to $1$, then the state is invalid.

If we `measure' a qubit in the horizontal direction we are using the basis vectors $(\begin{bmatrix}
    \frac{1}{\sqrt{2}}\\
    -\frac{1}{\sqrt{2}}
\end{bmatrix}, \begin{bmatrix}
    \frac{1}{\sqrt{2}}\\
    \frac{1}{\sqrt{2}}
\end{bmatrix})$.
Some of the values that we're going to repeatedly run into may be familiar from an earlier education in trigonometry.\\
$\sin(45\degree) = \frac{1}{\sqrt{2}}\\
\cos(45\degree) = \frac{1}{\sqrt{2}}\\
\sin(30\degree) = \frac{1}{2} \qquad \cos(60\degree) = \frac{1}{2}\\
\cos(30\degree) = \frac{\sqrt{3}}{2} \qquad \sin(60\degree) = \frac{\sqrt{3}}{2}$\\

\begin{note}
    We cannot distinguish between $\ket{v}$ and $-\ket{v}$.\\
    $\ket{v} = a\ket{0} + b\ket{1}$ has the same probabilities involved as $-\ket{v} = -a\ket{0} - b\ket{1}$ (being $a^2$ for $\ket{0}$ and $b^2$ for $\ket{1}$).
\end{note}

If we were to rotate our observation appparatus (or perspective) by $\Theta\degree$, the new basis vectors would be $(\begin{bmatrix}
    \cos(\Theta)\\
    -\sin(\Theta)
\end{bmatrix}, \begin{bmatrix}
    \sin(\Theta)\\
    \cos(\Theta)
\end{bmatrix})$.

% \subsection{Sub Section 3}
% \label{sub_sec:sub_section_3}

% \begin{theorem}
% This is a theorem.
% \end{theorem}
% \begin{proof}
% This is a proof.
% \end{proof}
% \begin{example}
% This is an example.
% \end{example}
% \begin{explanation}
% This is an explanation.
% \end{explanation}
% \begin{claim}
% This is a claim.
% \end{claim}
% \begin{corollary}
% This is a corollary.
% \end{corollary}
% \begin{prop}
% This is a proposition.
% \end{prop}
% \begin{lemma}
% This is a lemma.
% \end{lemma}
% \begin{question}
% This is a question.
% \end{question}
% \begin{solution}
% This is a solution.
% \end{solution}
% \begin{exercise}
% This is an exercise.
% \end{exercise}
% \begin{definition}[Definition]
% This is a definition.
% \end{definition}
% \begin{note}
% This is a note.
% \end{note}

% subsection sub_section_2 (end)

\newpage