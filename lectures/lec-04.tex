\lesson{Tue, 7 February 2023, 2:50pm -- 4:10pm}{Week 3, Tuesday}

Qubits are represented by unit vectors.

\begin{definition}
    A vector is \emph{orthonormal} if it is both a unit vector and orthogonal.
\end{definition}

\begin{example}
    The following are the basis vectors for $R^3$:
    $\begin{bmatrix}
        1\\
        0\\
        0
    \end{bmatrix}
    \begin{bmatrix}
        0\\
        1\\
        0
    \end{bmatrix}
    \begin{bmatrix}
        0\\
        0\\
        1
    \end{bmatrix}$

    This means that any vector can be written as a linear combination of these basis vectors.

    If we were talking about spin, for example, we could use $R^2$, with the following basis vectors:\\
    $\ket{\uparrow} = \begin{bmatrix}
        1\\
        0
    \end{bmatrix}\\
    \ket{\downarrow} = \begin{bmatrix}
        0\\
        1
    \end{bmatrix}\\
    \ket{\rightarrow} = \begin{bmatrix}
        \frac{1}{\sqrt{2}}\\
        \frac{1}{\sqrt{2}}
    \end{bmatrix}\\
    \ket{\leftarrow} = \begin{bmatrix}
        \frac{1}{\sqrt{2}}\\
        \frac{1}{\sqrt{2}}
    \end{bmatrix}$

    Meaning,\\
    $
    \ket{\nearrow} = \ket{\uparrow} \cdot \ket{\rightarrow} = \begin{bmatrix}
        \frac{1}{2}\\
        \frac{-\sqrt{3}}{2}
    \end{bmatrix}\\
    \ket{\swarrow} = \ket{\downarrow} \cdot \ket{\leftarrow} \begin{bmatrix}
        \frac{\sqrt{3}}{2}\\
        \frac{1}{2}
    \end{bmatrix}
    $.
\end{example}

\begin{definition}
    A \emph{matrix is a rectangular array of numbers.}\\
    The `gates' that are the primary components of quantum computing algorithms correspond to matricies.
\end{definition}

\begin{definition}
    To \emph{transpose} a matrix means to `rotate' it so that its rows become columns, and its columns become rows.
\end{definition}

\begin{example}
    $M = \begin{bmatrix}
        1 & 4\\
        2 & 5\\
        3 & 6
    \end{bmatrix}$\\
    $M$ has $3$ rows by $2$ columns.

    $M^T = \begin{bmatrix}
        1 & 2 & 3\\
        4 & 5 & 6
    \end{bmatrix}$\\
    $M^T$ has $2$ rows by $3$ columns.
\end{example}

If multiplying two matricies, one with dimensions $n \times m$ and the other with dimensions $m \times p$, then the result will have dimensions $n \times p$ (ie, the resultant matrix's dimensions will be number of the first matrix's rows, by the number of the second matrix's columns).

\begin{example}
    $
    \begin{bmatrix}
        1 & 4\\
        2 & 5\\
        3 & 6
    \end{bmatrix}
    \cdot
    \begin{bmatrix}
        -2 & 1 & 3\\
        4 & 7 & -2
    \end{bmatrix}
    =
    \begin{bmatrix}
        14 & 29 & -5\\
        16 & 37 & -4\\
        18 & 45 & -3
    \end{bmatrix}
    $
\end{example}

\begin{definition}
    An \emph{identity matrix} is a matrix when, multiplied with another matrix, simply yields that other matrix.
\end{definition}

\begin{example}
    $
    \begin{bmatrix}
        14 & 29 & -5\\
        16 & 37 & -4\\
        18 & 45 & -3
    \end{bmatrix}
    \cdot
    \begin{bmatrix}
        1 & 0 & 0\\
        0 & 1 & 0\\
        0 & 0 & 1
    \end{bmatrix}
    =
    \begin{bmatrix}
        14 & 29 & -5\\
        16 & 37 & -4\\
        18 & 45 & -3
    \end{bmatrix}
    $
\end{example}

\begin{note}
    If a matrix, $H$, were multiplied by its transpose (ie, $H \cdot H^T$), and if the multiplication were to yield an identity matrix, $I$, then it is orthogonal. In other words, if $H \cdot H^T = I$, then $H$ is orthogonal. When dealing with complex numbers, this is called a \emph{unitary matrix} instead, rather than an `orthogonal matrix'.
\end{note}

\begin{definition}
    A \emph{tensor product} (represented as $A \otimes B$) can be found by, for each value in the left-hand side, multiplying said value with all of the values on the right-hand side. This operation can be referred to as calculating the \emph{Kronecker product} or \emph{matrix direct product}, when dealing specifically with matricies as opposed to other tensors.\\
    A \emph{tensor} is a generalisation of matricies; where a matrix has two dimensions (rows and columns), a tensor can have any number of dimensions.
\end{definition}

\begin{example}
    $
    \begin{bmatrix}
        2\\
        3
    \end{bmatrix}
    \otimes
    \begin{bmatrix}
        -1\\
        4\\
        7
    \end{bmatrix}
    =
    \begin{bmatrix}
        -2\\
        8\\
        14\\
        -3\\
        12\\
        21
    \end{bmatrix}
    $
\end{example}

The state of a quantum system corresponds to a vector.
The state of a quantum system is a tensor product of qubits.
A vector is separable if it can be written as a tensor product of two other vecotrs.
A vector is entangled if it \emph{cannot} be written as the tensor product of two other vectors.

% \subsection{Sub Section 3}
% \label{sub_sec:sub_section_3}

% \begin{theorem}
% This is a theorem.
% \end{theorem}
% \begin{proof}
% This is a proof.
% \end{proof}
% \begin{example}
% This is an example.
% \end{example}
% \begin{explanation}
% This is an explanation.
% \end{explanation}
% \begin{claim}
% This is a claim.
% \end{claim}
% \begin{corollary}
% This is a corollary.
% \end{corollary}
% \begin{prop}
% This is a proposition.
% \end{prop}
% \begin{lemma}
% This is a lemma.
% \end{lemma}
% \begin{question}
% This is a question.
% \end{question}
% \begin{solution}
% This is a solution.
% \end{solution}
% \begin{exercise}
% This is an exercise.
% \end{exercise}
% \begin{definition}[Definition]
% This is a definition.
% \end{definition}
% \begin{note}
% This is a note.
% \end{note}

% subsection sub_section_2 (end)

\newpage