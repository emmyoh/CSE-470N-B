\lesson{Thu, 23 February 2023, 2:50pm -- 4:10pm}{Week 5, Thursday}

\begin{note}
    A system of $n$ qubits will be in $C^{2^n}$ space; ie, if you preform tensor multiplication on $n$ qubits (each being in $C^2$ space--meaning each vector has two elements), you will get $C^{2^n}$ space--meaning the resultant vector will have $2^n$ elements.
\end{note}

\begin{note}
    If two qubits are \emph{not} entangled, then we can examine each one independently. Additionally, we can represent this state both as a vector in $C^4$ space, and as the tensor product of two vectors in $C^2$ space.
\end{note}

Suppose we wanted to perform tensor multiplication with groupings; to do so, we would multiply as we would any other algebraic expression: $\ket{vw} = (a \ket{0} + b \ket{1}) \otimes (c \ket{0} + d \ket{1}) = (ac \ket{00} + ad \ket{01} + bc \ket{10} + bd \ket{11})$, where $\ket{0} = \begin{bmatrix}
    1\\
    0
\end{bmatrix}$ and $\ket{1} = \begin{bmatrix}
    0\\
    1
\end{bmatrix}$.

\begin{note}
    $\ket{v} \otimes \ket{w} = \ket{v} \ket{w} = \ket{vw} \neq \braket{v}{w}$.
\end{note}

The standard basis vectors for $C^4 = C^2 \otimes C^2$ space are $\ket{00}$, $\ket{01}$, $\ket{10}$, and $\ket{11}$. This is equivalent to $\begin{bmatrix}
    1\\
    0\\
    0\\
    0
\end{bmatrix}$, $\begin{bmatrix}
    0\\
    1\\
    0\\
    0
\end{bmatrix}$, $\begin{bmatrix}
    0\\
    0\\
    1\\
    0
\end{bmatrix}$, and $\begin{bmatrix}
    0\\
    0\\
    0\\
    1
\end{bmatrix}$.

% \subsection{Sub Section 3}
% \label{sub_sec:sub_section_3}

% \begin{theorem}
% This is a theorem.
% \end{theorem}
% \begin{proof}
% This is a proof.
% \end{proof}
% \begin{example}
% This is an example.
% \end{example}
% \begin{explanation}
% This is an explanation.
% \end{explanation}
% \begin{claim}
% This is a claim.
% \end{claim}
% \begin{corollary}
% This is a corollary.
% \end{corollary}
% \begin{prop}
% This is a proposition.
% \end{prop}
% \begin{lemma}
% This is a lemma.
% \end{lemma}
% \begin{question}
% This is a question.
% \end{question}
% \begin{solution}
% This is a solution.
% \end{solution}
% \begin{exercise}
% This is an exercise.
% \end{exercise}
% \begin{definition}[Definition]
% This is a definition.
% \end{definition}
% \begin{note}
% This is a note.
% \end{note}

% subsection sub_section_2 (end)

\newpage