\lesson{Thu, 16 March 2023, 2:50pm -- 4:10pm}{Week 8, Thursday}

As a reminder, the Pauli-X gate (`NOT' gate) is $X = \begin{bmatrix}
    0 & 1 \\
    1 & 0\\
\end{bmatrix}$, the Pauli-Y gate is $Y = \begin{bmatrix}
    0 & -1 \\
    1 & 0\\
\end{bmatrix}$, and the Pauli-Z gate is $Z = \begin{bmatrix}
    1 & 0 \\
    0 & -1\\
\end{bmatrix}$.

The Pauli gates perform the following operations on the basis states:\\
\begin{itemize}
    \item $X(a \ket{0} + b \ket{1}) = b \ket{0} + a \ket{1}$
    \item $Y(a \ket{0} + b \ket{1}) = b \ket{0} - a \ket{1}$
    \item $Z(a \ket{0} + b \ket{1}) = a \ket{0} - b \ket{1}$
\end{itemize}

If one were to pass the following inputs into Bell's Circuit ($B$), these would be the outputs:
\begin{itemize}
    \item $B(\ket{00}) = \frac{1}{\sqrt{2}} \ket{00} + \frac{1}{\sqrt{2}} \ket{11}$
    \item $B(\ket{01}) = \frac{1}{\sqrt{2}} \ket{01} + \frac{1}{\sqrt{2}} \ket{10}$
    \item $B(\ket{10}) = \frac{1}{\sqrt{2}} \ket{00} - \frac{1}{\sqrt{2}} \ket{11}$
    \item $B(\ket{11}) = \frac{1}{\sqrt{2}} \ket{01} - \frac{1}{\sqrt{2}} \ket{10}$
\end{itemize}

\begin{definition}
    \emph{Quantum communication} is the exploitation of entanglement to transmit information.
\end{definition}
\subsection{Quantum Communication Protocols}
\label{sub_sec:quantum_communication_protocols}
\begin{itemize}
    \item Superdense coding
        \begin{itemize}
            \item To communicate a number of classical bits by use of a smaller number of qubits. In other words, using one qubit to send two classical bits.
            \item Start with two entangled qubits in superposition, one sent to Alice, and the other sent to Bob.
            \item Alice wants to send two classical bits to Bob; she can send either $00$, $01$, $10$, or $11$---essentially, she is going to send Bob one qubit.
            \item If she wants to send $00$, she does nothing (she applies the identity gate, $I$). The quantum state is unchanged, being $\frac{1}{\sqrt{2}} \ket{00} + \frac{1}{\sqrt{2}} \ket{11}$.
            \item If she wants to send $01$, she applies the Pauli-X (or `NOT') gate, $X$. The quantum state is now $\frac{1}{\sqrt{2}} \ket{10} + \frac{1}{\sqrt{2}} \ket{01}$.
            \item If she wants to send $10$, she applies the Pauli-Z gate, $Z$. The quantum state is now $\frac{1}{\sqrt{2}} \ket{00} - \frac{1}{\sqrt{2}} \ket{11}$.
            \item If she wants to send $11$, she applies the Pauli-Y gate, $Y$. The quantum state is now $\frac{1}{\sqrt{2}} \ket{01} - \frac{1}{\sqrt{2}} \ket{10}$.
            \item The quantum states that result from Alice's transformations match our expected outputs of Bell's circuit. Knowing this, Bob now applies the reverse Bell circuit, $B^{\dagger}$, to the qubit he received from Alice. This will give him the two classical bits he wanted to receive.
        \end{itemize}
    \item Quantum teleportation
        \begin{itemize}
            \item Sending two classical bits to transmit (`teleport') one qubit.
            \item Bob and Alice share an entangled pair, and Alice has another qubit of her own in some state, $a \ket{0} + b \ket{1}$. She wants to send Bob her qubit, but she cannot clone it, so she must `cut-and-paste' it (she will lose it, but Bob will have it).
            \item Alice entangles her two qubits, entangling all three qubits. She first applies the Hadamard gate, with the quantum state becoming $(a \ket{0} + b \ket{1}) \otimes (\frac{1}{\sqrt{2}} \ket{00} + \frac{1}{\sqrt{2}} \ket{11}) = \frac{a}{\sqrt{2}} \ket{000} + \frac{a}{\sqrt{2}} \ket{011} + \frac{b}{\sqrt{2}} \ket{100} + \frac{b}{\sqrt{2}} \ket{111} = \frac{a}{\sqrt{2}} \ket{00} \otimes \ket{0} + \frac{a}{\sqrt{2}} \ket{01} \otimes \ket{1} + \frac{b}{\sqrt{2}} \ket{10} \otimes \ket{0} + \frac{b}{\sqrt{2}} \ket{11} \otimes \ket{1}$. After applying the CNOT gate to her qubits, the state becomes $\frac{a}{\sqrt{2}} \ket{00} \otimes \ket{0} + \frac{a}{\sqrt{2}} \ket{01} \otimes \ket{1} + \frac{b}{\sqrt{2}} \ket{11} \otimes \ket{0} + \frac{b}{\sqrt{2}} \ket{10} \otimes \ket{11} = \frac{a}{\sqrt{2}} \ket{0} \otimes \ket{0} \otimes \ket{0} + \frac{a}{\sqrt{2}} \ket{0} \otimes \ket{1} \otimes \ket{1} + \frac{b}{\sqrt{2}} \ket{1} \otimes \ket{1} \otimes \ket{0} + \frac{b}{\sqrt{2}} \ket{1} \otimes \ket{0} \otimes \ket{11}$. The state of the qubit Alice wants to send to Bob is now entangled with the two qubits she shared with Bob.
            \item She then applies the Hadamard gate to the qubit which she wants to send, and the state becomes $\frac{a}{\sqrt{2}}(\frac{1}{\sqrt{2}} \ket{0} + \frac{1}{\sqrt{2}} \ket{1}) \otimes \ket{0} \otimes \ket{0} + \frac{a}{\sqrt{2}}(\frac{1}{\sqrt{2}} \ket{0} + \frac{1}{\sqrt{2}} \ket{1}) \otimes \ket{1} \otimes \ket{1} + \frac{b}{\sqrt{2}} (\frac{1}{\sqrt{2}} \ket{0} - \frac{1}{\sqrt{2}} \ket{1}) \otimes \ket{1} \otimes \ket{0} + \frac{b}{\sqrt{2}}(\frac{1}{\sqrt{2}} \ket{0} - \frac{1}{\sqrt{2}} \ket{1}) \otimes \ket{0} \otimes \ket{1} = \frac{1}{2} \ket{00} \otimes (a \ket{0} + b\ket{1}) + \frac{1}{2} \ket{01} \otimes (a \ket{1} + b \ket{0}) + \frac{1}{2} \ket{10} \otimes (a \ket{0} - b \ket{1}) + \frac{1}{2} \ket{11} \otimes (a \ket{1} - b \ket{0})$.
            \item If Alice measures $\ket{00}$, Bob's state is $a \ket{0} + b \ket{1}$. If Alice measures $\ket{01}$, Bob's state is $a \ket{1} + b \ket{0}$. If Alice measures $\ket{10}$, Bob's state is $a \ket{0} - b \ket{1}$. If Alice measures $\ket{11}$, Bob's state is $a \ket{1} - b \ket{1}$.
            \item Alice now sends Bob the two classical bits she measured (ie, sends $00$ if $\ket{00}$, $10$ if $\ket{10}$, etc). If Bob gets $00$, he does nothing (he applies the identity gate, $I$); his state is $a \ket{0} + b\ket{1}$. If Bob gets $01$, he applies the Pauli-X gate, $X$; his state is $a \ket{0} + b \ket{1}$. If Bob gets $10$, he applies the Pauli-Z gate, $Z$. If Bob gets $11$, he applies the Pauli-Y gate, $Y$.
        \end{itemize}
\end{itemize}
% subsection quantum_communication_protocols (end)

% \subsection{Sub Section 3}
% \label{sub_sec:sub_section_3}

% \begin{theorem}
% This is a theorem.
% \end{theorem}
% \begin{proof}
% This is a proof.
% \end{proof}
% \begin{example}
% This is an example.
% \end{example}
% \begin{explanation}
% This is an explanation.
% \end{explanation}
% \begin{claim}
% This is a claim.
% \end{claim}
% \begin{corollary}
% This is a corollary.
% \end{corollary}
% \begin{prop}
% This is a proposition.
% \end{prop}
% \begin{lemma}
% This is a lemma.
% \end{lemma}
% \begin{question}
% This is a question.
% \end{question}
% \begin{solution}
% This is a solution.
% \end{solution}
% \begin{exercise}
% This is an exercise.
% \end{exercise}
% \begin{definition}[Definition]
% This is a definition.
% \end{definition}
% \begin{note}
% This is a note.
% \end{note}

% subsection sub_section_2 (end)

\newpage