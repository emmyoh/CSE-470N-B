\lesson{Tue, 14 February 2023, 2:50pm – 4:10pm}{Week 4, Tuesday}

\begin{definition}
    A \emph{qubit} is a unit vector (`ket') in $C^2$.\
    When we measure a qubit, we are, in effect, choosing a direction for measurement. Which actually means that we are choosing an orthonormal basis vector.
\end{definition}

\begin{note}
    To represent classical bits, we can do so with the equation, $\ket{v} = x \cdot \ket{b_1} + y \cdot \ket{b_2}$ where $x^2 + y^2 = 1$.\\
    We will always write it so that $\ket{b_1} = 0$ and that $\ket{b_2} = 1$, not the other way around; the order in which we write our basis vectors conveys that the first will represent a $0$, and that the second will represent a $1$.
\end{note}

\begin{figure}[h]
    \centering
    \begin{adjustbox}{width=\textwidth}
    \begin{tikzpicture}[shorten >=1pt,node distance=3.5cm,on grid,auto]
        \node[draw,rectangle] (a)   {Filter \#1 ($0^{\degree}$)};
        \node[] (a_left) [left=of a] {$\ket{v} = \frac{1}{\sqrt{2}}\cdot\begin{bmatrix}0\\1\end{bmatrix} + \frac{1}{\sqrt{2}}\cdot\begin{bmatrix}1\\0\end{bmatrix}$};
        \node[below of=a] (a_below) {Blocked $\begin{bmatrix}0\\1\end{bmatrix}$};

        \node[draw,rectangle,right of=a] (b)   {Filter \#2 ($45^{\degree})$};
        \node[below of=b] (b_below) {$(\begin{bmatrix}\frac{1}{\sqrt{2}}\\-\frac{1}{\sqrt{2}}\end{bmatrix} \begin{bmatrix}\frac{1}{\sqrt{2}}\\\frac{1}{\sqrt{2}}\end{bmatrix})$};
        
        \node[draw,rectangle,right of=b] (c)   {Filter \#3 ($90^{\degree})$};
        \coordinate[right of=c] (c_right);
        \path[-latex,shorten >= 2pt, shorten <= 2pt]
            (a_left)
                edge node {} (a)
            (a)
                edge node {} (a_below)
                edge node {Through $\begin{bmatrix}1\\0\end{bmatrix}$} (b)           
            (b)
                edge node {} (b_below)
                edge node {$\begin{bmatrix}\frac{1}{\sqrt{2}}\\-\frac{1}{\sqrt{2}}\end{bmatrix}$} (c)
            (c)
                edge node {$\begin{bmatrix}0\\1\end{bmatrix}$} (c_right);
    \end{tikzpicture}
    \end{adjustbox}
    \caption{Three repeated filters, blocking photons. The probabilities became a certain outcome.}\label{fig:experiment4}
\end{figure}

\newpage

\begin{definition}
    When two waves collide and destroy one another, that is \emph{destructive interference}. When the two waves combine, that is \emph{constructive interference}.
\end{definition}

\begin{example}
    Find $a$ and $b$ when $\ket{v} = a \cdot \begin{bmatrix}
        1\\
        0
    \end{bmatrix} + b \cdot \begin{bmatrix}
        0\\
        1
    \end{bmatrix}$, where $\ket{v}$ is the interaction between $\ket{←}$ and $\ket{→}$.\\
    $\ket{v} = \frac{1}{\sqrt{2}} \cdot \ket{←} + \frac{1}{\sqrt{2}} \cdot \ket{→} = \frac{1}{\sqrt{2}} \cdot \begin{bmatrix}\frac{1}{\sqrt{2}}\\\frac{1}{\sqrt{2}}\end{bmatrix} + \frac{1}{\sqrt{2}} \cdot \begin{bmatrix}\frac{1}{\sqrt{2}}\\-\frac{1}{\sqrt{2}}\end{bmatrix} = \begin{bmatrix}\frac{1}{2}\\\frac{1}{2}\end{bmatrix} + \begin{bmatrix}\frac{1}{2}\\-\frac{1}{2}\end{bmatrix} = \begin{bmatrix}1\\0\end{bmatrix}$.\\
    Therefore, $a = 1$, $b = 0$. $\ket{←}$ and $\ket{→}$, both with their own probabilities, constructively interfered to achieve one certain outcome.
\end{example}


% \subsection{Sub Section 3}
% \label{sub_sec:sub_section_3}

% \begin{theorem}
% This is a theorem.
% \end{theorem}
% \begin{proof}
% This is a proof.
% \end{proof}
% \begin{example}
% This is an example.
% \end{example}
% \begin{explanation}
% This is an explanation.
% \end{explanation}
% \begin{claim}
% This is a claim.
% \end{claim}
% \begin{corollary}
% This is a corollary.
% \end{corollary}
% \begin{prop}
% This is a proposition.
% \end{prop}
% \begin{lemma}
% This is a lemma.
% \end{lemma}
% \begin{question}
% This is a question.
% \end{question}
% \begin{solution}
% This is a solution.
% \end{solution}
% \begin{exercise}
% This is an exercise.
% \end{exercise}
% \begin{definition}[Definition]
% This is a definition.
% \end{definition}
% \begin{note}
% This is a note.
% \end{note}

% subsection sub_section_2 (end)

\newpage