\lesson{Tue, 28 February 2023, 2:50pm -- 4:10pm}{Week 6, Tuesday}

Two qubits are entangled if, when we measure (observe) one, the state of the other qubit changes instantaneously. The state of any quantum system corresponds to a unit vector in a vector space; that vector can be represented as a linear combination of basis vectors. If two qubits are \emph{not} entangled, then we can examine (measure) one qubit without affecting the state of the other---they are independent. Suppose that the basis vectors that we are using for the first qubit are $\ket{a_1}$ and $\ket{a_2}$, and the basis vectors that we are using for the second qubit are $\ket{b_1}$ and $\ket{b_2}$. Then the state of this system of two qubits is represented by a vector in $R^2$ like $r\ket{a_0 b_0} + s\ket{a_0 b_1} + t\ket{a_1 b_0} + u\ket{a_1 b_1}$. Since $r$, $s$, $t$, and $u$ are probability amplitudes, $r^2 + s^2 + t^2 + u^2 = 1$.

\begin{example}
    \textbf{Separable (not entangled; unentangled)}
    
    $\ket{v} = x_0\ket{a_0} + x_1\ket{a_1} \qquad x_0^2 +x_1^2 = 1$

    $\ket{w} = y_0\ket{b_0} + y_1\ket{b_1} \qquad y_0^2 +y_1^2 = 1$\\

    $\ket{v} \otimes \ket{w} = (x_0\ket{a_0} + x_1\ket{a_1}) \otimes (y_0\ket{b_0} + y_1\ket{b_1}) = x_0y_0\ket{a_0 b_0} + x_0y_1\ket{a_0 b_1} + x_1y_0\ket{a_1 b_0} + x_1y_1\ket{a_1 b_1}$\\

    The squared probability amplitudes sum up to $1$.\\
    $(x_0 y_0)^2 + (x_0 y_1)^2 + (x_1 y_0)^2 + (x_1 y_1)^2 = 1$\\
    To demonstrate this we can use the fact that $x_0^2 + x_1^2 = 1$ and $y_0^2 + y_1^2 = 1$.\\
    $x_0^2(y_0^2 + y_1^2) + x_1^2(y_0^2 + y_1^2) = 1$\\
    $x_0^2 + x_1^2 = 1$
\end{example}

If a vector is entangled, then it cannot be represented as the tensor product of two vectors, $\begin{bmatrix}
    a\\
    b
\end{bmatrix} \otimes \begin{bmatrix}
    c\\
    d
\end{bmatrix} = \begin{bmatrix}
    ac\\
    ad\\
    bc\\
    bd
\end{bmatrix}$.\\
A quick trick to determine if a vector in $C^4$ is entangled is to check if the product of the first and last elements is equal to the product of the middle two elements; if it is, then the vector is not entangled---it is separable. If the products are different, then the vector is entangled (ie, if $ac \cdot bd \neq ad \cdot bc$, then the vector is entangled).

\pagebreak

\begin{example}
    \textbf{Separable (not entangled; unentangled)}

    $(a\ket{0} + b\ket{1}) \otimes (c\ket{0} + d\ket{1}) = ac\ket{0 0} + ad\ket{0 1} + bc\ket{1 0} + bd\ket{1 1}$
    \begin{enumerate}
        \item What is the probability that the second qubit is $\ket{0}$?
        \item Now, let's measure the first qubit. Suppose we get $\ket{0}$. What is the probability that the second qubit is $\ket{0}$?
        \item Now, what is the probability that the second qubit is $\ket{0}$?
    \end{enumerate}
    \textbf{\emph{Solution}}
    \begin{enumerate}
        \item $(ac)^2 + (bc)^2 = a^2 c^2 + b^2 c^2 = c^2(a^2 + b^2) = c^2$.
        \item  $ac\ket{00} + ad\ket{01}$. This is an invalid quantum state; $(ac)^2 + (ad)^2 \neq 1$. We know this because we know that $(ac)^2 + (ad)^2 + (bc)^2 + (bd)^2 = 1$. We can normalise using the length, $\sqrt{(ac)^2 + (ad)^2} = \sqrt{a^2 c^2 + a^2 d^2} = \sqrt{a^2(c^2 + d^2)} = a$. $\frac{ac\ket{00}}{a} + \frac{ad\ket{01}}{a} = c\ket{00} + d\ket{01}$. This is a valid quantum state.
        \item $(c)^2 = c^2$. This is the same as the probability that the second qubit is $\ket{0}$ when we don't measure the first qubit. This means that the first qubit and the second qubit are independent. They are \emph{not} entangled.
    \end{enumerate}
\end{example}

\begin{example}
    \textbf{Entangled}

    $a\ket{00} + b\ket{01} + c\ket{10} + d\ket{11}$\\
    \begin{enumerate}
        \item What is the probability that the second qubit is $\ket{0}$?
        \item Now, let's measure the first qubit. Suppose we get $\ket{0}$. What is the probability that the second qubit is $\ket{0}$?
        \item Now, what is the probability that the second qubit is $\ket{0}$?
    \end{enumerate}
    \textbf{\emph{Solution}}
    \begin{enumerate}
        \item $a^2 + c^2$
        \item The system state becomes $a\ket{00} + b\ket{01}$. This is not a valid quantum state, because $(a)^2 + (b)^2 + (c)^2 + (d)^2 = 1$. We can normalise using the length, $\sqrt{a^2 + b^2}$. $\frac{a\ket{00}}{\sqrt{a^2 + b^2}} + \frac{b\ket{01}}{\sqrt{a^2 + b^2}}$. This is a valid quantum state.
        \item $\frac{a^2}{a^2 + b^2}$. This does not equal $a^2 + c^2$ (the probability that the second qubit is $\ket{0}$ when we don't measure the first qubit). This means that the first qubit and the second qubit are not independent. They are entangled; they both changed.
    \end{enumerate}
\end{example}

Hadamard gate: $\mathrm{H} = \frac{1}{\sqrt{2}}\begin{bmatrix}
    1 & 1\\
    1 & -1
\end{bmatrix} = \begin{bmatrix}
    \frac{1}{\sqrt{2}} & \frac{1}{\sqrt{2}}\\
    \frac{1}{\sqrt{2}} & -\frac{1}{\sqrt{2}}
\end{bmatrix}$.

CNot gate: $\mathrm{CNot} = \begin{bmatrix}
    1 & 0 & 0 & 0\\
    0 & 1 & 0 & 0\\
    0 & 0 & 0 & 1\\
    0 & 0 & 1 & 0
\end{bmatrix} \begin{bmatrix}
    a\\
    b\\
    c\\
    d
\end{bmatrix} = \begin{bmatrix}
    a\\
    b\\
    d\\
    c
\end{bmatrix}$.

% \subsection{Sub Section 3}
% \label{sub_sec:sub_section_3}

% \begin{theorem}
% This is a theorem.
% \end{theorem}
% \begin{proof}
% This is a proof.
% \end{proof}
% \begin{example}
% This is an example.
% \end{example}
% \begin{explanation}
% This is an explanation.
% \end{explanation}
% \begin{claim}
% This is a claim.
% \end{claim}
% \begin{corollary}
% This is a corollary.
% \end{corollary}
% \begin{prop}
% This is a proposition.
% \end{prop}
% \begin{lemma}
% This is a lemma.
% \end{lemma}
% \begin{question}
% This is a question.
% \end{question}
% \begin{solution}
% This is a solution.
% \end{solution}
% \begin{exercise}
% This is an exercise.
% \end{exercise}
% \begin{definition}[Definition]
% This is a definition.
% \end{definition}
% \begin{note}
% This is a note.
% \end{note}

% subsection sub_section_2 (end)

\newpage