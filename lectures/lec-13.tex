\lesson{Thu, 6 April 2023, 2:50pm -- 4:10pm}{Week 11, Thursday}

\begin{definition}
    \emph{Quantum parallelism} refers to the speedup that we experience with quantum algorithms, as opposed to classical algorithms, which results from the fact that we put our input into a superposition. A quantum algorithm is designed to manipulate these superpositions to get useful results.
\end{definition}

\begin{definition}
    $P$ is the set of problems for which there is a classical algorithm that can solve it in polynomial time.
    $NP$ is the set of problems that can be verified by a classical algorithm in polynomial time.
\end{definition}

The typical steps in a quantum algorithm are as follows:
\begin{enumerate}
    \item Begin with qubits in a particular classical state
    \item Put the system into a superposition of many states
    \item Act on the superposition with several quantum gates
\end{enumerate}

\begin{definition}
    \emph{Deutsch's algorithm} attempts to minimise the query complexity when faced with the following problem:\\

    Given a function $f: \{0, 1\} \rightarrow \{0, 1\}$, is this function constant or balanced?\\
    
    Possible functions could be:
    \begin{itemize}
        \item Constant - 0 ($f(0)=0$, $f(1)=0$)
        \item Constant - 1 ($f(0)=1$, $f(1)=1$)
        \item Identity ($f(0)=0$, $f(1)=1$)
        \item Swap ($f(0)=1$, $f(1)=0$)
    \end{itemize}

    Instead of requiring both $f(0)$ and $f(1)$ to determine the nature of $f$, Deutsch's algorithm requires only one.\\
    The input to the algorithm is the function, $f$. The algorithm represents the possible $f$ functions with quantum gates:

    Constant - 0:\\
    $\begin{bmatrix}
        1 & 1\\
        0 & 0
    \end{bmatrix} \cdot \ket{0} = \ket{0}$\\

    $\begin{bmatrix}
        1 & 1\\
        0 & 0
    \end{bmatrix} \cdot \ket{1} = \ket{0}$\\


    Constant - 1:\\
    $\begin{bmatrix}
        0 & 0\\
        1 & 1
    \end{bmatrix} \cdot \ket{0} = \ket{1}$\\

    $\begin{bmatrix}
        0 & 0\\
        1 & 1
    \end{bmatrix} \cdot \ket{1} = \ket{1}$\\


    Identity:\\
    $\begin{bmatrix}
        1 & 0\\
        0 & 1
    \end{bmatrix} \cdot \ket{0} = \ket{0}$\\

    $\begin{bmatrix}
        1 & 0\\
        0 & 1
    \end{bmatrix} \cdot \ket{1} = \ket{1}$\\


    Swap:\\
    $\begin{bmatrix}
        0 & 1\\
        1 & 0
    \end{bmatrix} \cdot \ket{0} = \ket{1}$\\

    $\begin{bmatrix}
        0 & 1\\
        1 & 0
    \end{bmatrix} \cdot \ket{1} = \ket{0}$\\        
\end{definition}

\begin{figure}[ht]
    \centering
    \begin{quantikz}
        \lstick{$\ket{0}$\\$x$}\slice{A} & \gate{H}\slice{B} & \gate[wires=2]{F}\slice{C} & \gate{H}\slice{D} & \meter{} & \qw \\
        \lstick{$\ket{1}$\\$y$} & \gate{H} & \qw & \qw & \qw & \qw
    \end{quantikz}
    \caption{The circuit for Deutsch's algorithm, where $F(\ket{xy}) = F(\ket{x}, \ket{y}) = \ket{x} \otimes (\ket{y} \oplus f(\ket{x}))$; $F$ is the controlled version of the quantum gate representation of the function, $f$, whichever of the four functions it may be. The state of the system at A is $\ket{01}$, $\frac{1}{2}(\ket{00}-\ket{01}+\ket{10}-\ket{11})$ at B, $\frac{1}{2}(\ket{0} \otimes f(0) - \ket{0} \otimes \overline{f(0)} + \ket{1} \otimes f(1) - \ket{1} \otimes \overline{f(1)})$ at C, and either $\ket{0}$ or $\ket{1}$ at D.}\label{fig:lec13fig1}
\end{figure}

\pagebreak

% \subsection{Sub Section 3}
% \label{sub_sec:sub_section_3}

% \begin{theorem}
% This is a theorem.
% \end{theorem}
% \begin{proof}
% This is a proof.
% \end{proof}
% \begin{example}
% This is an example.
% \end{example}
% \begin{explanation}
% This is an explanation.
% \end{explanation}
% \begin{claim}
% This is a claim.
% \end{claim}
% \begin{corollary}
% This is a corollary.
% \end{corollary}
% \begin{prop}
% This is a proposition.
% \end{prop}
% \begin{lemma}
% This is a lemma.
% \end{lemma}
% \begin{question}
% This is a question.
% \end{question}
% \begin{solution}
% This is a solution.
% \end{solution}
% \begin{exercise}
% This is an exercise.
% \end{exercise}
% \begin{definition}[Definition]
% This is a definition.
% \end{definition}
% \begin{note}
% This is a note.
% \end{note}

% subsection sub_section_2 (end)

\newpage