
\lesson{Tue, 31 January 2023, 2:50pm – 4:10pm}{Week 2, Tuesday}

\begin{figure}[h]
    \centering
    \begin{tikzpicture}[shorten >=4pt,node distance=3cm,on grid,auto]
        \node[draw,rectangle] (a)   {Up-down};
        \node[] (a_left) [left=of a] {};
        \coordinate[below of=a] (a_below);

        \node[draw,rectangle,right of=a] (b)   {Up-down};
        \coordinate[below of=b] (b_below);
        \coordinate[right of=b] (b_right);
        \path[-latex,shorten >= 2pt, shorten <= 2pt]
            (a_left)
                edge node {} (a)
            (a)
                edge node {↓ $50\%$} (a_below)
                edge node {↑ $50\%$} (b)           
            (b)
                edge node {↓ $100\%$} (b_below)
                edge node {↑ $0\%$} (b_right);
    \end{tikzpicture}
    \caption{Two repeated detectors of whether a particle's spin is up or down. The same property is being measured; the percentages heading into the second detector are known.}
    \label{fig:experiment2}
\end{figure}

\begin{figure}[h]
    \centering
    \begin{tikzpicture}[shorten >=4pt,node distance=3cm,on grid,auto]
        \node[draw,rectangle] (a)   {Up-down};
        \node[] (a_left) [left=of a] {};
        \coordinate[below of=a] (a_below);

        \node[draw,rectangle,right of=a] (b)   {Left-right};
        \coordinate[below of=b] (b_below);
        
        \node[draw,rectangle,right of=b] (c)   {Up-down};
        \coordinate[below of=c] (c_below);
        \coordinate[right of=c] (c_right);
        \path[-latex,shorten >= 2pt, shorten <= 2pt]
            (a_left)
                edge node {} (a)
            (a)
                edge node {↓ $50\%$} (a_below)
                edge node {↑ $50\%$} (b)           
            (b)
                edge node {← $50\%$} (b_below)
                edge node {→ $50\%$} (c)
            (c)
                edge node {↓ $50\%$} (c_below)
                edge node {↑ $50\%$} (c_right);
    \end{tikzpicture}
    \caption{Three repeated detectors, now detecting three different properties of a particle. Now, the percentages for spin up or down are not known; the first and last detectors are probabilistically independent because of the middle detector.}
    \label{fig:experiment3}
\end{figure}

`To talk about an electron that is both spin up and spin right is nonsensical.'

\begin{definition}
    \emph{Tunneling} is a phenomenon in quantum mechanics where an object on the quantum scale can penetrate barriers in a manner that's contradictory to what classical mechanics predicts. In other words, an object can sometimes move through something that should seemingly stop its movement.
\end{definition}

\begin{definition}
    \emph{Quantum decoherence} is when the wave function that describes the quantum state of a particle `collapses' (ie, the quantum state can no longer be predicted or described by the wave function). With decoherence, information about the system is lost into the environment; if a quantum system were perfectly isolated (ie, if nothing could interact with it), it would maintain coherence indefinitely.
\end{definition}

With $n$ qubits, a quantum algorithm can search up to $2^n$ states simultaneously. This is the advantage of quantum computers–modelling complex systems and searching through a large set of possibilities is where quantum computers can be useful.

% \subsection{Sub Section 2}
% \label{sub_sec:sub_section_2}

% \begin{theorem}
% This is a theorem.
% \end{theorem}
% \begin{proof}
% This is a proof.
% \end{proof}
% \begin{example}
% This is an example.
% \end{example}
% \begin{explanation}
% This is an explanation.
% \end{explanation}
% \begin{claim}
% This is a claim.
% \end{claim}
% \begin{corollary}
% This is a corollary.
% \end{corollary}
% \begin{prop}
% This is a proposition.
% \end{prop}
% \begin{lemma}
% This is a lemma.
% \end{lemma}
% \begin{question}
% This is a question.
% \end{question}
% \begin{solution}
% This is a solution.
% \end{solution}
% \begin{exercise}
% This is an exercise.
% \end{exercise}
% \begin{definition}[Definition]
% This is a definition.
% \end{definition}
% \begin{note}
% This is a note.
% \end{note}

% subsection sub_section_2 (end)

\newpage