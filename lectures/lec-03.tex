\lesson{Thu, 2 February 2023, 2:50pm – 4:10pm}{Week 2, Thursday}

`The state of a quantum system corresponds to a vector in a vector space of complex numbers.'

\begin{definition}
  A \emph{vector} is a list of numbers.
  The length (or magnitude; denoted by $\lvert\ket{v}\rvert$) of a vector can be found by calculating the square root of the sum of squares of the horizontal and vertical components.
  Scalar multiplication can be performed by multiplying every value of a vector by the scalar.
  Vector addition can be performed by adding every element in a vector with the element in the corresponding position in another vector.
  Vector multiplication (also referred to as finding a dot product, or an inner product; denoted by the product $\braket{v}{w}$) can be done between two vectors with the same dimensions. If the result of the multiplication is $0$, the two vectors are \emph{orthogonal}.
\end{definition}
\begin{example}
  Row: $\bra{v} = \begin{bmatrix}
    2 & 3 & 4
  \end{bmatrix}$\\
  Column: $\ket{w} = 
  \begin{bmatrix}
    2\\
    3\\
    4
  \end{bmatrix}
  $\\
  Magnitude: $\ket{v} = 
  \begin{bmatrix}
    a\\
    b\\
  \end{bmatrix}$\\
  $\lvert\ket{w}\rvert = \sqrt{a^2 + b^2}$\\
  Scalar multiplication: $\ket{v} = 
  \begin{bmatrix}
    3\\
    -2
  \end{bmatrix}$\\
  $4 \cdot \ket{v} = 4 \cdot
  \begin{bmatrix}
    3\\
    -2
  \end{bmatrix} = \begin{bmatrix}
    12\\
    -8
  \end{bmatrix}$\\
  Vector addition: $\begin{bmatrix}
    1\\
    2\\
    3
  \end{bmatrix} + \begin{bmatrix}
    7\\
    -3\\
    4
  \end{bmatrix} = \begin{bmatrix}
    8\\
    -1\\
    7
  \end{bmatrix}$\\
  Multiplication: $\begin{bmatrix}
    2 & 3 & 4
  \end{bmatrix} \cdot \begin{bmatrix}
    -1\\
    2\\
    7
  \end{bmatrix} = -2 + 6 + 28 = 32$\\
\end{example}

\begin{note}
  For the purposes of this course, we must be able to find the length of a vector, preform scalar multiplication, perform vector addition, and check for orthogonality.
\end{note}

\begin{definition}
  A set of \emph{basis vectors} is a set of vectors that can be combined in a linear combination to make any other vector in the vector space.
\end{definition}
\begin{example}
  Possible basis vectors with two dimensions:\\
  $\begin{bmatrix}
    76.9513\\
    \pi
  \end{bmatrix} = 76.9513 \cdot \begin{bmatrix}
    1\\
    0
  \end{bmatrix} - \pi \cdot \begin{bmatrix}
    0\\
    1
  \end{bmatrix}$\\
  Possible basis vectors with three dimensions:\\
  $\begin{bmatrix}
    1\\
    0\\
    0
  \end{bmatrix} \begin{bmatrix}
    0\\
    1\\
    0
  \end{bmatrix} \begin{bmatrix}
    0\\
    0\\
    1
  \end{bmatrix}$
\end{example}

\begin{note}
  $\ket{→} = \begin{bmatrix}
    \frac{1}{\sqrt{2}}\\
    -\frac{1}{\sqrt{2}}
  \end{bmatrix}$\\
  $\ket{←} = \begin{bmatrix}
    \frac{1}{\sqrt{2}}\\
    \frac{1}{\sqrt{2}}
  \end{bmatrix}$\\
  $\ket{↗} = \begin{bmatrix}
    \frac{1}{2}\\
    \frac{-\sqrt{3}}{2}
  \end{bmatrix}$\\
  $\ket{↙} = \begin{bmatrix}
    \frac{\sqrt{3}}{2}\\
    \frac{1}{2}
  \end{bmatrix}$
\end{note}

% \subsection{Sub Section 3}
% \label{sub_sec:sub_section_3}

% \begin{theorem}
% This is a theorem.
% \end{theorem}
% \begin{proof}
% This is a proof.
% \end{proof}
% \begin{example}
% This is an example.
% \end{example}
% \begin{explanation}
% This is an explanation.
% \end{explanation}
% \begin{claim}
% This is a claim.
% \end{claim}
% \begin{corollary}
% This is a corollary.
% \end{corollary}
% \begin{prop}
% This is a proposition.
% \end{prop}
% \begin{lemma}
% This is a lemma.
% \end{lemma}
% \begin{question}
% This is a question.
% \end{question}
% \begin{solution}
% This is a solution.
% \end{solution}
% \begin{exercise}
% This is an exercise.
% \end{exercise}
% \begin{definition}[Definition]
% This is a definition.
% \end{definition}
% \begin{note}
% This is a note.
% \end{note}

% subsection sub_section_2 (end)

\newpage