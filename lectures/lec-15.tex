\lesson{Thu, 20 April 2023, 2:50pm -- 4:10pm}{Week 13, Thursday}

\begin{definition}
    \emph{Simon's periodicity algorithm} is a probabilistic algorithm which involves a quantum and a classical part; it solves the following problem:\\

    Given a function $f: \{0, 1\}^n \rightarrow \{0, 1\}^n$ that is periodic, find its period. A period function is $f(x) = f(x \oplus c)$, where $c$ is the period.
\end{definition}

\begin{figure}[ht]
    \centering
    \begin{quantikz}
        \lstick{$\ket{0}^n$\\$x$} & \qw\qwbundle{n} & \gate{H^{\otimes n}} & \qw\qwbundle{n} & \gate[wires=2]{F} & \qw\qwbundle{n} & \gate{H^{\otimes n}} & \qw\qwbundle{n} & \meter{} & \qw \\
        \lstick{$\ket{0}^n$\\$y$} & \qw\qwbundle{n} & \qw & \qw & \qw & \qw\qwbundle{n} & \qw & \qw & \qw & \qw
    \end{quantikz}
    \caption{The general circuit for Simon's periodicity algorithm.}\label{fig:lec15fig1}
\end{figure}

\begin{figure}[ht]
    \centering
    \begin{quantikz}
        \lstick{$\ket{00}$\\$x$}\slice{A} & \qw\qwbundle{2} & \gate{H^{\otimes 2}}\slice{B} & \qw\qwbundle{2} & \gate[wires=2]{F}\slice{C} & \qw\qwbundle{n} & \gate{H^{\otimes 2}}\slice{D} & \qw\qwbundle{2} & \meter{} & \qw \\
        \lstick{$\ket{00}$\\$y$} & \qw\qwbundle{2} & \qw & \qw & \qw & \qw\qwbundle{2} & \qw & \qw & \qw & \qw
    \end{quantikz}
    \caption{The circuit for Simon's periodicity algorithm when $n = 2$, where $F(x_1x_2, y_1y_2) = \ket{x_1x_2} \otimes (y_1y_2 \oplus f(x_1x_2)) = \ket{x_1x_2} \otimes f(x_1x_2)$.}\label{fig:lec15fig2}
\end{figure}

If $\ket{x}$ is an answer that you get from Simon's algorithm, it is \emph{always} true that $x \cdot c = 0$, where $c$ is the period and $x$ is a series of classical bits.\\

Suppose we run the algorithm, when $n = 5$, and we get an answer of $1 0 1 0 0$. This means that $1 \cdot c_0 \oplus 0 \cdot c_1 \oplus 1 \cdot c_2 \oplus 0 \cdot c_3 \oplus 0 \cdot c_4 = 0$, $c_0 \oplus c_2 = 0$, so $c_0 = c_2$. Suppose we run the algorithm again, and get $0 0 1 0 0$; $0 \cdot c_0 \oplus 0 \cdot c_1 \oplus 1 \cdot c_2 \oplus 0 \cdot c_3 \oplus 0 \cdot c_4 = 0 \oplus 0 \oplus c_2 \oplus 0 \oplus 0 = 0$, meaning $c_2 = 0$. If $c_2 = 0$ and $c_0 = c_2$, then $c_0 = 0$. Suppose we run the algorithm a third time, and get $1 1 1 1 0$; $c_0 \oplus c_1 \oplus c_2 \oplus c_3 = 0 = 0 \oplus c_1 \oplus 0 \oplus c_3 = 0 = c_1 \oplus c_3 = 0 = c_1 = 3$. Suppose we run the algorithm a fourth time and get $0 0 1 1 1$; we know that $c_2 \oplus c_3 \oplus c_4 = 0 = c_3 \oplus c_4 = 0$, meaning $c = 0 1 0 1 1$.

\begin{definition}
    \emph{Grover's search algorithm} is an algorithm that solves the following problem:\\

    Given a function that takes, as input, $n$ qubits, and selects one of them, find the value selected.

    \begin{equation*}
        f(x)=\begin{cases}
                  1 \quad &\text{if} \, x \neq x_0 \\
                  0 \quad &\text{if} \, x = x_0 \\
             \end{cases}
    \end{equation*}
\end{definition}

% \subsection{Sub Section 3}
% \label{sub_sec:sub_section_3}

% \begin{theorem}
% This is a theorem.
% \end{theorem}
% \begin{proof}
% This is a proof.
% \end{proof}
% \begin{example}
% This is an example.
% \end{example}
% \begin{explanation}
% This is an explanation.
% \end{explanation}
% \begin{claim}
% This is a claim.
% \end{claim}
% \begin{corollary}
% This is a corollary.
% \end{corollary}
% \begin{prop}
% This is a proposition.
% \end{prop}
% \begin{lemma}
% This is a lemma.
% \end{lemma}
% \begin{question}
% This is a question.
% \end{question}
% \begin{solution}
% This is a solution.
% \end{solution}
% \begin{exercise}
% This is an exercise.
% \end{exercise}
% \begin{definition}[Definition]
% This is a definition.
% \end{definition}
% \begin{note}
% This is a note.
% \end{note}

% subsection sub_section_2 (end)

\newpage