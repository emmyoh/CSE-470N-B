\lesson{Thu, 9 March 2023, 2:50pm -- 4:10pm}{Week 7, Thursday}

All quantum operations (except measurements) are reversible and are represented by unitary matrices.

\begin{definition}
    An operation is \emph{reversible} if, from the output, we can determine the inputs.
\end{definition}

\begin{definition}
    A \emph{unitary matrix} is a matrix whose inverse is its \emph{conjugate} transpose. An \emph{orthogonal matrix} is a matrix whose inverse is its transpose. See \textbf{Definition 4.12} and \textbf{Definition 4.13} from \textbf{Lecture 4}.
\end{definition}

The behaviour of the CNOT gate can be represented with the following two tables:\\
\begin{tabular}{l|l}
    Input & Output\\
    \hline
    $x \ y$ & $x \ x \oplus y$\\
    $0 \ 0$ & $0 \ 0$\\
    $0 \ 1$ & $0 \ 1$\\
    $1 \ 0$ & $1 \ 1$\\
    $1 \ 1$ & $1 \ 0$
\end{tabular}
\begin{tabular}{l|l}
    Input & Output\\
    \hline
    $\ket{00}$ & $\ket{00}$\\
    $\ket{01}$ & $\ket{01}$\\
    $\ket{10}$ & $\ket{11}$\\
    $\ket{11}$ & $\ket{10}$
\end{tabular}\\

CNOT:\\
$\begin{bmatrix}
    1 & 0 & 0 & 0\\
    0 & 1 & 0 & 0\\
    0 & 0 & 0 & 1\\
    0 & 0 & 1 & 0
\end{bmatrix} \begin{bmatrix}
    a\\
    b\\
    c\\
    d
\end{bmatrix} = \begin{bmatrix}
    a\\
    b\\
    d\\
    c
\end{bmatrix}$

\begin{example}
    $\mathrm{CNOT}(r\ket{00} + s\ket{01} + t\ket{10} + u\ket{11}) = r\ket{00} + s\ket{01} + u\ket{10} + t\ket{11}$. The probabilitiy amplitudes for $\ket{10}$ and $\ket{11}$ are flipped.
\end{example}

\pagebreak

\begin{explanation}
    CNOT is a unitary matrix:\\
    $\begin{bmatrix}
        1 & 0 & 0 & 0\\
        0 & 1 & 0 & 0\\
        0 & 0 & 0 & 1\\
        0 & 0 & 1 & 0
    \end{bmatrix} \begin{bmatrix}
        1 & 0 & 0 & 0\\
        0 & 1 & 0 & 0\\
        0 & 0 & 0 & 1\\
        0 & 0 & 1 & 0
    \end{bmatrix} = \begin{bmatrix}
        1 & 0 & 0 & 0\\
        0 & 1 & 0 & 0\\
        0 & 0 & 1 & 0\\
        0 & 0 & 0 & 1
    \end{bmatrix}$\\
    It is its own inverse and its own conjugate transpose.
\end{explanation}

\begin{figure}[h]
    \centering
    \begin{tikzpicture}[shorten >=1pt,node distance=3.5cm,on grid,auto]
        \node[] (start_top) {$\frac{1}{\sqrt{2}}\ket{0} + \frac{1}{\sqrt{2}}\ket{1}$};
        \node[right of=start_top,circle,draw=black,fill=black] (end_top) {};
        \coordinate[right of=end_top] (beyond_top);
        \node[below of=start_top] (start_bottom) {$\ket{0}$};
        \node[right of=start_bottom] (end_bottom) {$\oplus$};
        \coordinate[right of=end_bottom] (beyond_bottom);
        
        \path[-,shorten >= 2pt, shorten <= 2pt]
            (start_top)
                edge node {} (end_top)
            (end_top)
                edge node {} (end_bottom)
                edge node {} (beyond_top)
            (start_bottom)
                edge node {} (end_bottom)
            (end_bottom)
                edge node {} (beyond_bottom);
    \end{tikzpicture}
    \caption{$(\frac{1}{\sqrt{2}}\ket{0} + \frac{1}{\sqrt{2}}\ket{1}) \otimes \ket{0} = \frac{1}{\sqrt{2}}\ket{00} + \frac{1}{\sqrt{2}}\ket{10} = \frac{1}{\sqrt{2}}\begin{bmatrix}
        1\\
        0\\
        0\\
        0
    \end{bmatrix} \begin{bmatrix}
        0\\
        0\\
        1\\
        0
    \end{bmatrix} = \frac{1}{\sqrt{2}}\begin{bmatrix}
        1\\
        0\\
        1\\
        0
    \end{bmatrix}$}\label{fig:lec10fig1}
\end{figure}

$\frac{1}{\sqrt{2}}\begin{bmatrix}
    1\\
    0\\
    0\\
    1
\end{bmatrix} = \frac{1}{\sqrt{2}}\begin{bmatrix}
    1\\
    0\\
    0\\
    0
\end{bmatrix} + \frac{1}{\sqrt{2}}\begin{bmatrix}
    0\\
    0\\
    0\\
    1
\end{bmatrix} = \frac{1}{\sqrt{2}}\ket{00} + \frac{1}{\sqrt{2}}\ket{11}$.

Quantum gates with one input:
\begin{enumerate}
    \item Pauli's gate
    \item Identity gate ($\begin{bmatrix}
        1 & 0\\
        0 & 1
    \end{bmatrix}$)
    \item Z gate ($\begin{bmatrix}
        1 & 0\\
        0 & -1
    \end{bmatrix}$; relative phase shift)
    \item X gate ($\begin{bmatrix}
        0 & 1\\
        1 & 0
    \end{bmatrix}$; flips the probability amplitudes)
    \item Y gate ($\begin{bmatrix}
        0 & -i\\
        i & 0
    \end{bmatrix}$)
    \item Hadamard gate ($\frac{1}{\sqrt{2}}\begin{bmatrix}
        1 & 1\\
        1 & -1
    \end{bmatrix}$)
\end{enumerate}
All of the above gates are reversible, are equal to their own transpose (except for the Y gate), and yield the identity gate when multiplied by their transpose.

\begin{example}
    $\mathrm{z}(a\ket{0} + b\ket{1}) = a\ket{0} - b\ket{1}$
\end{example}

\begin{note}
    In the above example, we are observing a \emph{relative phase shift}.
\end{note}

\pagebreak

\begin{figure}[ht]
    \centering
    \begin{tikzpicture}[shorten >=1pt,node distance=3.5cm,on grid,auto]
        \node[] (start_top) {$\ket{0}$};
        \node[right of=start_top,draw] (h) {$H$};
        \node[right of=h,circle,draw=black,fill=black] (end_top) {};
        \coordinate[right of=end_top] (beyond_top);
        \node[below of=start_top] (start_bottom) {$\ket{0}$};
        \node[below of=end_top] (end_bottom) {$\oplus$};
        \coordinate[right of=end_bottom] (beyond_bottom);
        
        \path[-]
            (start_top)
                edge node {} (h)
            (h)
                edge node {$\frac{1}{\sqrt{2}}\ket{0} + \frac{1}{\sqrt{2}}\ket{1}$} (end_top)
            (end_top)
                edge node {} (end_bottom)
                edge node {} (beyond_top)
            (start_bottom)
                edge node {} (end_bottom)
            (end_bottom)
                edge node {} (beyond_bottom);
    \end{tikzpicture}
    \caption{An example of Bell's Circuit, with an input of $\ket{00}$.}\label{fig:lec10fig2}
\end{figure}

At the start of the circuit, the state of the system is $\ket{0} \otimes \ket{0} = \ket{00}$. After passing through the Hadamard gate, the state of the system is $(\frac{1}{\sqrt{2}} \ket{0} + \frac{1}{\sqrt{2}} \ket{1}) \otimes \ket{0} = \frac{1}{\sqrt{2}}\ket{00} + \frac{1}{\sqrt{2}} \ket{10} = \frac{1}{\sqrt{2}} \begin{bmatrix}
    1\\
    0\\
    0\\
    0\\
\end{bmatrix} + \frac{1}{\sqrt{2}} \begin{bmatrix}
    0\\
    0\\
    1\\
    0
\end{bmatrix} = \frac{1}{\sqrt{2}} \begin{bmatrix}
    1\\
    0\\
    1\\
    0
\end{bmatrix}$. Lastly, after passing through the CNOT gate, the state of the system is $\frac{1}{\sqrt{2}} \begin{bmatrix}
    1\\
    0\\
    0\\
    1
\end{bmatrix} = \frac{1}{\sqrt{2}}\begin{bmatrix}
    1\\
    0\\
    0\\
    0\\
\end{bmatrix} + \frac{1}{\sqrt{2}} \begin{bmatrix}
    0\\
    0\\
    0\\
    1
\end{bmatrix} = \frac{1}{\sqrt{2}} \ket{00} + \frac{1}{\sqrt{2}} \ket{11} = \mathrm{B}(\ket{00})$, which is an entangled state.\\

There is a pattern to Bell's Circuit:\\
$\mathrm{B}(\ket{00}) = \frac{1}{\sqrt{2}} \ket{00} + \frac{1}{\sqrt{2}} \ket{11}$\\
$\mathrm{B}(\ket{01}) = \frac{1}{\sqrt{2}} \ket{01} + \frac{1}{\sqrt{2}} \ket{10}$\\
$\mathrm{B}(\ket{10}) = \frac{1}{\sqrt{2}} \ket{00} - \frac{1}{\sqrt{2}} \ket{11}$\\
$\mathrm{B}(\ket{11}) = \frac{1}{\sqrt{2}} \ket{01} - \frac{1}{\sqrt{2}} \ket{10}$\\

\begin{exercise}
    \textbf{Question}: What are the standard basis vectors for $R^4$?\\

    \textbf{Solution}:\\
    $\begin{bmatrix}
        1\\
        0\\
        0\\
        0
    \end{bmatrix}$
    $\begin{bmatrix}
        0\\
        1\\
        0\\
        0
    \end{bmatrix}$
    $\begin{bmatrix}
        0\\
        0\\
        1\\
        0
    \end{bmatrix}$
    $\begin{bmatrix}
        0\\
        0\\
        0\\
        1
    \end{bmatrix}$\\
    $\ket{00}$, $\ket{01}$, $\ket{10}$, $\ket{11}$
\end{exercise}

If one were to pass an input through Bell's Circuit, and then to the reverse Bell's Circuit, the output would be the same as the input.

% \subsection{Sub Section 3}
% \label{sub_sec:sub_section_3}

% \begin{theorem}
% This is a theorem.
% \end{theorem}
% \begin{proof}
% This is a proof.
% \end{proof}
% \begin{example}
% This is an example.
% \end{example}
% \begin{explanation}
% This is an explanation.
% \end{explanation}
% \begin{claim}
% This is a claim.
% \end{claim}
% \begin{corollary}
% This is a corollary.
% \end{corollary}
% \begin{prop}
% This is a proposition.
% \end{prop}
% \begin{lemma}
% This is a lemma.
% \end{lemma}
% \begin{question}
% This is a question.
% \end{question}
% \begin{solution}
% This is a solution.
% \end{solution}
% \begin{exercise}
% This is an exercise.
% \end{exercise}
% \begin{definition}[Definition]
% This is a definition.
% \end{definition}
% \begin{note}
% This is a note.
% \end{note}

% subsection sub_section_2 (end)

\newpage