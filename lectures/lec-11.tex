\lesson{Tue, 14 March 2023, 2:50pm -- 4:10pm}{Week 8, Tuesday}

\begin{definition}
    The \emph{no cloning theorem} states that it is not possible to \underline{copy} an arbitrary quantum state without destroying the original. In other words, it is possible to `cut and paste', but it is not possible to `copy and paste'. To develop an intuitive understanding of why this is, suppose we have a qubit in a superposition, $a \ket{0} + b \ket{1}$, and we wished to copy it---if we wished to measure the qubit to copy it, the superposition collapses.
\end{definition}

Suppose we had a gate to copy a qubit, $G$.
If we were to supply it an input to copy and a qubit to copy the first qubit onto, we would observe the following outputs:\\
$G(\ket{00}) = \ket{00}$\\
$G(\ket{10}) = \ket{11}$\\
$G(\frac{1}{\sqrt{2}} \ket{0} + \frac{1}{\sqrt{2}} \ket{1} \otimes \ket{0}) = (\frac{1}{\sqrt{2}} \ket{0} + \frac{1}{\sqrt{2}} \ket{1}) \otimes (\frac{1}{\sqrt{2}} \ket{0} + \frac{1}{\sqrt{2}} \ket{1})$\\
$G(\frac{1}{\sqrt{2}} \ket{00} + \frac{1}{\sqrt{2}} \ket{10}) = \frac{1}{\sqrt{2}} \ket{00} + \frac{1}{\sqrt{2}} \ket{01} + \frac{1}{\sqrt{2}} \ket{10} + \frac{1}{\sqrt{2}} \ket{11}$\\
$G(\frac{1}{\sqrt{2}} \ket{00} + \frac{1}{\sqrt{2}} \ket{10}) = G(\frac{1}{\sqrt{2}}\ket{00}) + G(\frac{1}{\sqrt{2}}\ket{10}) = \frac{1}{\sqrt{2}} G(\ket{00}) + \frac{1}{\sqrt{2}} G(\ket{10}) = \frac{1}{\sqrt{2}} \ket{00} + \frac{1}{\sqrt{2}} \ket{11}$\\
As you can see, the last two outputs are not the same, despite the same input; this is a contradiction. The contradiction demonstrates that such a gate cannot exist.
\begin{figure}[ht]
    \centering
    \begin{tikzpicture}[shorten >=1pt,node distance=3.5cm,on grid,auto]
        \node[] (start_top) {$\ket{0}$};
        \node[right of=start_top,draw] (h) {$H$};
        \node[right of=h,circle,draw=black,fill=black] (end_top) {};
        \coordinate[right of=end_top] (beyond_top);
        \node[below of=start_top] (start_bottom) {$\ket{0}$};
        \node[below of=end_top] (end_bottom) {$\oplus$};
        \coordinate[right of=end_bottom] (beyond_bottom);
        
        \path[-]
            (start_top)
                edge node {\textcolor{DarkGreen}{$\ket{00} = \ket{0} \otimes \ket{0}$}} (h)
            (h)
                edge node {$\frac{1}{\sqrt{2}}\ket{0} + \frac{1}{\sqrt{2}}\ket{1}$} (end_top)
                edge node[below] {\textcolor{DarkGreen}{\begin{adjustbox}{width=3cm}$(\frac{1}{\sqrt{2}}\ket{0} + \frac{1}{\sqrt{2}}\ket{1}) \otimes \ket{0} = \frac{1}{\sqrt{2}} \ket{00} + \frac{1}{\sqrt{2}} \ket{10} = \frac{1}{\sqrt{2}} \begin{bmatrix}
                    1\\
                    0\\
                    1\\
                    0
                \end{bmatrix}$\end{adjustbox}}} (end_top)
            (end_top)
                edge node {} (end_bottom)
                edge node {\begin{adjustbox}{width=3cm}$\frac{1}{\sqrt{2}}\begin{bmatrix}
                    1\\
                    0\\
                    0\\
                    0
                \end{bmatrix} + \frac{1}{\sqrt{2}}\begin{bmatrix}
                    0\\
                    0\\
                    1\\
                    0
                \end{bmatrix} = \frac{1}{\sqrt{2}}\begin{bmatrix}
                    1\\
                    0\\
                    1\\
                    0
                \end{bmatrix}$\end{adjustbox}} (beyond_top)
            (start_bottom)
                edge node {} (end_bottom)
            (end_bottom)
                edge node {\textcolor{DarkGreen}{\begin{adjustbox}{width=3.5cm}$\frac{1}{\sqrt{2}}\begin{bmatrix}
                    1\\
                    0\\
                    0\\
                    1
                \end{bmatrix} = \frac{1}{\sqrt{2}}\ket{00} + \frac{1}{\sqrt{2}}\ket{11} = \frac{1}{\sqrt{2}}\begin{bmatrix}
                    1\\
                    0\\
                    0\\
                    0
                \end{bmatrix} + \frac{1}{\sqrt{2}}\begin{bmatrix}
                    0\\
                    0\\
                    0\\
                    1
                \end{bmatrix}$\end{adjustbox}}} (beyond_bottom);
    \end{tikzpicture}
    \caption{An example of Bell's Circuit, with an input of $\ket{00}$. When you have a circuit of unentangled qubits, qubits on parallel lines are represented by a tensor product.}\label{fig:lec11fig1}
\end{figure}

`Sequential gates are represented by the product of matrices'.

The `C' in `C-NOT' stands for `controlled'---it is a `controlled operation'.\\
The `NOT' gate:\\
$\begin{bmatrix}
    0 & 1\\
    1 & 0
\end{bmatrix}$.\\
\begin{claim}
    The `C-NOT' gate acts analagously to an \verb+if+ statement. If the first qubit is $1$, then apply the `NOT' gate to the second qubit. In other words, if the $1^{\underline{\text{st}}}$ qubit is $0$, the second qubit remains the same and the second qubit is flipped.
\end{claim}

\begin{example}
    Suppose we had a hypothetical gate which looked like:\\
    $\begin{bmatrix}
        a & b\\
        c & d
    \end{bmatrix}$. Let's say we wished to create a `controlled' version of this gate, called $A$:\\
    $A = \begin{bmatrix}
        1 & 0 & 0 & 0\\
        0 & 1 & 0 & 0\\
        0 & 0 & a & b\\
        0 & 0 & c & d
    \end{bmatrix}$.\\
    $A(\ket{00}) = \ket{00}$, $A(\ket{01}) = \ket{01}$, $A(\ket{10}) = \ket{1} \otimes \begin{bmatrix}
        a\\
        c
    \end{bmatrix}$, $A(\ket{11}) = \ket{1} \otimes \begin{bmatrix}
        b\\
        d
    \end{bmatrix}$.\\
\end{example}

The `Taffoli' gate has two controls bits and one output:
\[ \begin{bmatrix}
    1 & 0 & 0 & 0 & 0 & 0 & 0 & 0\\
    0 & 1 & 0 & 0 & 0 & 0 & 0 & 0\\
    0 & 0 & 1 & 0 & 0 & 0 & 0 & 0\\
    0 & 0 & 0 & 1 & 0 & 0 & 0 & 0\\
    0 & 0 & 0 & 0 & 1 & 0 & 0 & 0\\
    0 & 0 & 0 & 0 & 0 & 1 & 0 & 0\\
    0 & 0 & 0 & 0 & 0 & 0 & 0 & 1\\
    0 & 0 & 0 & 0 & 0 & 0 & 1 & 0
\end{bmatrix} \]
We can think of this gate as the `C-C-NOT' gate (a `controlled'-`controlled'-`NOT' gate). It is reversible, and universal in the context of classical computing (just like `NAND').\\

\pagebreak

\begin{figure}[ht]
    \centering
    \begin{tikzpicture}[shorten >=1pt,node distance=3.5cm,on grid,auto]
        \node[] (start_top) {$\ket{x}$};
        \node[right of=start_top,circle,draw=black,fill=black] (end_top) {};
        \coordinate[right of=end_top] (beyond_top);

        \node[below of=start_top] (start_middle) {$\ket{y}$};
        \node[below of=end_top,circle,draw=black,fill=black] (end_middle) {};
        \coordinate[right of=end_middle] (beyond_middle);
        
        \node[below of=start_middle] (start_bottom) {$\ket{z}$};
        \node[below of=end_middle] (end_bottom) {$\oplus$};
        \coordinate[right of=end_bottom] (beyond_bottom);
        
        \path[-]
            (start_top)
                edge node {} (end_top)
            (end_top)
                edge node {} (end_bottom)
                edge node {$\ket{x}$} (beyond_top)
            (start_middle)
                edge node {} (end_middle)
            (end_middle)
                edge node {$\ket{y}$} (beyond_middle)
            (start_bottom)
                edge node {} (end_bottom)
            (end_bottom)
                edge node {$\ket{z \oplus (x \wedge y)}$} (beyond_bottom);
    \end{tikzpicture}
    \caption{A circuit demonstrating the Taffoli gate.}\label{fig:lec11fig2}
\end{figure}

\begin{note}
    There is a relationship between the two sides of the exclusive `or' operation ($\oplus$):\\
    $0 \oplus 0 = 0$, $0 \oplus 1 = 1$, $0 \oplus x = x$.\\
    $1 \oplus 0 = 1$, $1 \oplus 1 = 0$, $1 \oplus x = \overline{x}$.
\end{note}

\begin{question}
    Are there any quantum gates that are universal?
\end{question}

\hfill

\begin{solution}
    No.
\end{solution}

\begin{question}
    How many 1-qubit quantum gates are there?
\end{question}

\hfill

\begin{solution}
    Infinity.
\end{solution}

The `Fredkin' gate has one control bit, two outputs:
\[ \begin{bmatrix}
    1 & 0 & 0 & 0 & 0 & 0 & 0 & 0\\
    0 & 1 & 0 & 0 & 0 & 0 & 0 & 0\\
    0 & 0 & 1 & 0 & 0 & 0 & 0 & 0\\
    0 & 0 & 0 & 1 & 0 & 0 & 0 & 0\\
    0 & 0 & 0 & 0 & 1 & 0 & 0 & 0\\
    0 & 0 & 0 & 0 & 0 & 0 & 1 & 0\\
    0 & 0 & 0 & 0 & 0 & 1 & 0 & 0\\
    0 & 0 & 0 & 0 & 0 & 0 & 0 & 1
\end{bmatrix} \]
If $\ket{x} = 0$, $\ket{y}$ and $\ket{z}$ are unchanged. If $\ket{x} = \ket{1}$, $\ket{y^{\prime}}$ = $\ket{z}$, $\ket{z^{\prime}} = \ket{y}$.

\begin{figure}[ht]
    \centering
    \begin{tikzpicture}[shorten >=1pt,node distance=3.5cm,on grid,auto]
        \node[] (start_top) {$\ket{x}$};
        \node[right of=start_top,circle,draw=black,fill=black] (end_top) {};
        \coordinate[right of=end_top] (beyond_top);

        \node[below of=start_top] (start_middle) {$\ket{y}$};
        \node[below of=end_top,cross out,draw=black] (end_middle) {};
        \coordinate[right of=end_middle] (beyond_middle);
        
        \node[below of=start_middle] (start_bottom) {$\ket{z}$};
        \node[below of=end_middle,cross out,draw=black] (end_bottom) {};
        \coordinate[right of=end_bottom] (beyond_bottom);
        
        \path[-]
            (start_top)
                edge node {} (end_top)
            (end_top)
                edge node {} (end_middle)
                edge node {$\ket{x}$} (beyond_top)
            (start_middle)
                edge node {} (end_middle)
            (end_middle)
                edge node {} (end_bottom)
                edge node {$\ket{y^{\prime}}$} (beyond_middle)
            (start_bottom)
                edge node {} (end_bottom)
            (end_bottom)
                edge node {$\ket{z^{\prime}}$} (beyond_bottom);
    \end{tikzpicture}
    \caption{A circuit demonstrating the Fredkin gate.}\label{fig:lec11fig3}
\end{figure}

% \subsection{Sub Section 3}
% \label{sub_sec:sub_section_3}

% \begin{theorem}
% This is a theorem.
% \end{theorem}
% \begin{proof}
% This is a proof.
% \end{proof}
% \begin{example}
% This is an example.
% \end{example}
% \begin{explanation}
% This is an explanation.
% \end{explanation}
% \begin{claim}
% This is a claim.
% \end{claim}
% \begin{corollary}
% This is a corollary.
% \end{corollary}
% \begin{prop}
% This is a proposition.
% \end{prop}
% \begin{lemma}
% This is a lemma.
% \end{lemma}
% \begin{question}
% This is a question.
% \end{question}
% \begin{solution}
% This is a solution.
% \end{solution}
% \begin{exercise}
% This is an exercise.
% \end{exercise}
% \begin{definition}[Definition]
% This is a definition.
% \end{definition}
% \begin{note}
% This is a note.
% \end{note}

% subsection sub_section_2 (end)

\newpage