\lesson{Thu, 2 March 2023, 2:50pm -- 4:10pm}{Week 6, Thursday}

\begin{definition}
    The \emph{Copenhagen interpretation}, put simply, is that the wave function is a description of the state of the universe, and that the universe is in a superposition of states; in other words, everything is existing in all possible states at once. It's a broad term describing the interpretation of quantum mechanics shared by Niels Bohr and Werner Heisenberg---the term comes from the work that both performed together at the University of Copenhagen.
\end{definition}

\begin{definition}
    The \emph{EPR Paradox} is a thought experiment that was first proposed by Albert Einstein, Boris Podolsky, and Nathan Rosen in 1935. The paper describing the paradox argued that quantum mechanics was `incomplete', and that it was impossible to reconcile the theory with special relativity. The paradox is based on the idea that two particles, $A$ and $B$, are entangled, and that the measurement of the spin of particle $A$ will instantaneously affect the spin of particle $B$. This is a violation of the principle of locality, which states that the speed of light is the fastest possible speed of information transfer.
\end{definition}

\begin{definition}
    The \emph{Bell Inequality} is a mathematical inequality that was first proposed by John Stewart Bell in 1964. The inequality is a response to the EPR Paradox, and it describes an experiment that, if performed, would show whether or not the Copenhagen interpretation is correct or not.
    
    % $\frac{1}{\sqrt{2}} \ket{00} + \frac{1}{\sqrt{2}} \ket{11} = \frac{1}{\sqrt{2}}(\ket{0} \otimes \ket{0}) + \frac{1}{\sqrt{2}}(\ket{1} \otimes \ket{1}) = \frac{1}{\sqrt{2}} (\begin{bmatrix}
    %     1\\
    %     0
    % \end{bmatrix} \otimes \begin{bmatrix}
    %     1\\
    %     0
    % \end{bmatrix}) + \frac{1}{\sqrt{2}}(\begin{bmatrix}
    %     0\\
    %     1
    % \end{bmatrix} \otimes \begin{bmatrix}
    %     0\\
    %     1
    % \end{bmatrix}) = \frac{1}{\sqrt{2}} \begin{bmatrix}
    %     1\\
    %     0\\
    %     0\\
    %     0
    % \end{bmatrix} + \frac{1}{\sqrt{2}} \begin{bmatrix}
    %     0\\
    %     0\\
    %     0\\
    %     1
    % \end{bmatrix} = \frac{1}{\sqrt{2}} \begin{bmatrix}
    %     1\\
    %     0\\
    %     0\\
    %     1
    % \end{bmatrix} = \begin{bmatrix}
    %     \frac{1}{\sqrt{2}}\\
    %     0\\
    %     0\\
    %     \frac{1}{\sqrt{2}}
    % \end{bmatrix}$.\\

    Bell suggests performing the following experiment:
    \begin{enumerate}
        \item Create a stream of pairs of entangled particles.
        \item Measure and record the spin of one particle in each pair, using one randomly chosen direction out of three, $0^{\degree}$, $120^{\degree}$, $240^{\degree}$ ($\ket{\uparrow}$, $\ket{\downarrow}$, $\ket{\searrow}$, $\ket{\nwarrow}$, $\ket{\swarrow}$, $\ket{\nearrow}$)---the answer will be either $0$ or $1$.
        \item Measure the spin of the other particle in each pair, randomly choosing the direction again for the remaining particles.
        \item When all the pairs are measured, there will be two strings of $0$s and $1$s, one for each particle in each pair. If the pairs are the same in both strings (ie, if they `agree'), note that the outcome was $A$ in a new, third string---and if they are different (ie, if they `disagree'), note that the outcome was $D$.
        \item In the long run, the pairs will be the same (ie, the outcome will be $A$) about $\frac{1}{2}$ of the time, and different (ie, the outcome will be $D$) about $\frac{1}{2}$ of the time. Knowing that $P(A) = \frac{1}{2}$ and that $P(D) = \frac{1}{2}$, we know that the third, final string will be half $A$s and half $D$s.
    \end{enumerate}

    The sample space of this experiment would resemble the following:\\

    \begin{adjustbox}{width=\textwidth}
        \begin{tabular}{lll}
        $0^{\degree}$ & $120^{\degree}$ & $240^{\degree}$ \\
        $0$ & $0$ & $0$ \\
        $0$ & $0$ & $1$ \\
        $0$ & $1$ & $0$ \\
        $0$ & $1$ & $1$ \\
        $1$ & $0$ & $0$ \\
        $1$ & $0$ & $1$ \\
        $1$ & $1$ & $0$ \\
        $1$ & $1$ & $1$ \\
        \end{tabular}
        \begin{tabular}{|l|l|l|l|l|l|l|l|l|l|}
        \hline ($0^{\degree}$, $0^{\degree}$) & ($0^{\degree}$, $120^{\degree}$) & ($0^{\degree}$, $240^{\degree}$) & ($120^{\degree}$, $0^{\degree}$) & ($120^{\degree}$, $120^{\degree}$) & ($120^{\degree}$, $240^{\degree}$) & ($240^{\degree}$, $0^{\degree}$) & ($240^{\degree}$, $120^{\degree}$) & ($240^{\degree}$, $240^{\degree}$) & Number of $A$s \\
        \hline $A$ & $A$ & $A$ & $A$ & $A$ & $A$ & $A$ & $A$ & $A$ & $9$ \\
        \hline $A$ & $A$ & $D$ & $A$ & $A$ & $D$ & $D$ & $D$ & $A$ & $5$ \\
        \hline $A$ & $D$ & $A$ & $D$ & $A$ & $D$ & $A$ & $D$ & $A$ & $5$ \\
        \hline $A$ & $D$ & $D$ & $D$ & $A$ & $A$ & $D$ & $A$ & $A$ & $5$ \\
        \hline $A$ & $D$ & $D$ & $D$ & $A$ & $A$ & $D$ & $A$ & $A$ & $5$ \\
        \hline $A$ & $D$ & $A$ & $D$ & $A$ & $D$ & $A$ & $D$ & $A$ & $5$ \\
        \hline $A$ & $A$ & $D$ & $A$ & $A$ & $D$ & $D$ & $D$ & $A$ & $5$ \\
        \hline $A$ & $A$ & $A$ & $A$ & $A$ & $A$ & $A$ & $A$ & $A$ & $9$ \\
        \hline
        \end{tabular}
    \end{adjustbox}\\

    The sample space demonstrates, however, that the smallest possible proportion of $A$s is $\frac{5}{9}$. When performing the experiment in reality, the proportion of $A$s and $D$s will be closer to $\frac{1}{2}$ than $\frac{5}{9}$, validating the Copenhagen interpretation.
\end{definition}

% \subsection{Sub Section 3}
% \label{sub_sec:sub_section_3}

% \begin{theorem}
% This is a theorem.
% \end{theorem}
% \begin{proof}
% This is a proof.
% \end{proof}
% \begin{example}
% This is an example.
% \end{example}
% \begin{explanation}
% This is an explanation.
% \end{explanation}
% \begin{claim}
% This is a claim.
% \end{claim}
% \begin{corollary}
% This is a corollary.
% \end{corollary}
% \begin{prop}
% This is a proposition.
% \end{prop}
% \begin{lemma}
% This is a lemma.
% \end{lemma}
% \begin{question}
% This is a question.
% \end{question}
% \begin{solution}
% This is a solution.
% \end{solution}
% \begin{exercise}
% This is an exercise.
% \end{exercise}
% \begin{definition}[Definition]
% This is a definition.
% \end{definition}
% \begin{note}
% This is a note.
% \end{note}

% subsection sub_section_2 (end)

\newpage